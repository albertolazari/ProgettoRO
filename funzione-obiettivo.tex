La funzione obiettivo del modello può essere espressa in una forma generale come segue:
\begin{align*}
	\textrm{max} \quad & \textbf{Guadagno\ corsi\ boulder} + \textbf{Guadagno\ corsi\ lead} + \textbf{Guadagno\ corso\ agonisti} \\
	& + \textbf{Guadagno\ corso\ outdoor} - \textbf{Costi\ istruttori}
\end{align*}
dove i termini sono i seguenti:
\begin{flalign*}
	& \textbf{Guadagno\ corsi\ boulder} = \sum_{g \in G, o \in O} boulder_{g,o} \cdot guadagnoCorsoBoulder && \\
	& \textbf{Guadagno\ corsi\ lead} = \sum_{g \in G, o \in O} lead_{g,o} \cdot guadagnoCorsoLead && \\
	& \textbf{Guadagno\ corso\ agonisti} = \sum_{g \in G} t_g \cdot guadagnoCorsoAgonisti && \\
	& \textbf{Guadagno\ corso\ outdoor} = k \cdot guadagnoCorsoOutdoor &&
\end{flalign*}
\begin{flalign*}
	\textbf{Costi\ istruttori} = & \ costoIstruttore \cdot ( \textbf{Ore\ corsi\ individuali} + \textbf{Ore\ corso\ agonisti} && \\
	& + \textbf{Ore\ corso\ outdoor} ) &&
\end{flalign*}
\begin{align*}
	& \textbf{Ore\ corsi\ individuali} = \sum_{i \in I, g \in G, o \in O} b_{i,g,o} + l_{i,g,o} && \\
	& \textbf{Ore\ corso\ agonisti} = \sum_{i \in I, g \in G} 2 a_{i,g} && \\
	& \textbf{Ore\ corso\ outdoor} = \sum_{i \in I} d_i \cdot durataCorsoOutdoor &&
\end{align*}
\textbf{s.t.}

\begin{itemize}
	\item Ogni corso deve avere almeno un istruttore che lo segua:
\end{itemize}
\vspace*{-\baselineskip}
\begin{flalign*}
	& \sum_{i \in I} b_{i,g,o} \cdot M \geq boulder_{g,o} & \forall g \in G, \forall o \in O & \\
	& \sum_{i \in I} l_{i,g,o} \cdot M \geq lead_{g,o} & \forall g \in G, \forall o \in O & \\
	& \sum_{i \in I} a_{i,g} \geq t_{g} & \forall g \in G & \\
	& \sum_{i \in I} d_i \geq k &
\end{flalign*}

\begin{itemize}
	\item Un istruttore non può seguire contemporaneamente corsi di boulder e di lead:
\end{itemize}
\vspace*{-\baselineskip}
\begin{flalign*}
	& b_{i,g,o} + l_{i,g,o} \leq 1 & \forall i \in I, \forall g \in G, \forall o \in O &
\end{flalign*}

\begin{itemize}
	\item Un istruttore può seguire un numero massimo di corsi contemporaneamente:
\end{itemize}
\vspace*{-\baselineskip}
\begin{flalign*}
	& \sum_{i \in I} b_{i,g,o} \geq \frac{ boulder_{g,o} }{ maxCorsiIstruttore } & \forall g \in G, \forall o \in O & \\
	& \sum_{i \in I} l_{i,g,o} \geq \frac{ lead_{g,o} }{ maxCorsiIstruttore } & \forall g \in G, \forall o \in O &
\end{flalign*}

\begin{itemize}
	\item Un istruttore ha un numero massimo di ore settimanali che può mettere a disposizione:
\end{itemize}
\vspace*{-\baselineskip}
\begin{flalign*}
	& ( \sum_{g \in G, o \in O} b_{i,g,o} + l_{i,g,o} ) + ( \sum_{g \in G} 2 a_{i,g} ) + d_i \cdot durataCorsoOutdoor \leq maxOre_i & \forall i \in I &
\end{flalign*}

\begin{itemize}
	\item Se un istruttore sta seguendo il corso agonisti deve seguire solo quello, per tutte le 2 ore:
\end{itemize}
\vspace*{-\baselineskip}
\begin{flalign*}
	& b_{i,g,o} + b_{i,g,o + 1} + l_{i,g,o} + l_{i,g,o + 1} \leq ( 1 - a_{i,g} ) \cdot M & o = orarioCorsoAgonisti, \forall i \in I, \forall g \in G &
\end{flalign*}

\begin{itemize}
	\item Gli agonisti devono allenarsi almeno un certo numero di giorni alla settimana:
\end{itemize}
\vspace*{-\baselineskip}
\begin{flalign*}
	& \sum_{g \in G} t_g \geq minAllenamentiAgonisti &
\end{flalign*}
\begin{flalign*}
	& \textbf{Dominio: } \ b_{i,g,o}, l_{i,g,o}, t_{g}, a_{i,g}, k, d_i \in \{ 0, 1 \}
	& \forall i \in I, \forall g \in G, \forall o \in O &
\end{flalign*}
Si noti che non vengono fatte assunzioni sul numero di istruttori contemporanei per uno stesso costo, perché, non comportando nessun vantaggio la presenza di più istruttori in contemporanea, questo verrebbe limitato in tutti i casi a 1 dalla funzione obiettivo, nell'intento di minimizzare i costi degli istruttori per massimizzare il profitto. Questo effetto è garantito dalle variabili decisionali $t_g$ e $k$. \\
In uno scenario reale può essere utile mantenere la possibilità che due istruttori possano seguire uno stesso corso, invece di limitarli a 1 a priori, anche se questo non porterebbe a una soluzione ottima.
